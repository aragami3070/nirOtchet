\section{Фрагменты кода программы <<Вычисление факториала>>}\label{app:task1}

Функция очистки текстовых полей в форме:

\begin{minted}{c++}
	private: void ClearAll() { // очистка полей
		this->txtOutput->Text = "Ошибка ввода";
		errPr->SetError(txtInput, String::Empty);
	}
\end{minted}

Нажатие кнопки <<Вычислить>> установлено выполнение следующего кода:

\begin{minted}{c++}
	private: System::Void btnCalculate_Click(System::Object^ sender, System::EventArgs^ e) {
		ClearAll();
		long long InputNumber;
		// переводим сторку из TextBox в число
		bool result = Int64::TryParse(this->txtInput->Text, InputNumber);
		// ввели не число
		if (!result) {
			errPr->SetError(txtInput, "Введено не целое число");
		}
		else {
			if (InputNumber >= 20) {
				errPr->SetError(txtInput, "Слишком большое число");
			}
			else{
				// результат
				long long OutputNumber = fact(InputNumber);
				// отрицательное число
				if (OutputNumber == -1) {
					errPr->SetError(txtInput, "Введено отрицательное число");
				}
				// все нормально
				else {
					// записываем в поле вывода
					this->txtOutput->Text = System::Convert::ToString(OutputNumber);
				}
			}
		}
	}
\end{minted}

\section{Фрагменты кода программы <<Простые вычисления>>}\label{app:task2}

Нажатие кнопки <<Вычислить>> установлено выполнение следующего кода:

\begin{minted}{c++}
	private: System::Void button1_Click(System::Object^ sender, System::EventArgs^ e) {
		ClearAll();
		long long InputX;
		long long InputY;
		// переводим сторку из TextBox в число
		bool parseX = Int64::TryParse(this->txtInX->Text, InputX);
		bool parseY = Int64::TryParse(this->txtInY->Text, InputY);
		// ввели не число
		if (!parseX) {
			errPrX->SetError(txtInX, "Введено не целое число");
			if (!parseY) {
				errPrY->SetError(txtInY, "Введено не целое число");
			}
		}
		else if (!parseY) {
			errPrY->SetError(txtInY, "Введено не целое число");
		}
		// проверка на деление на ноль
		else {
			if ((InputX == 0 && InputY == 0) || 
				(InputX == -4 && abs(InputY) == 2)) {
				errPrX->SetError(txtInX, "Деление на ноль");
				errPrY->SetError(txtInY, "Деление на ноль");
			}
			else{
				// все нормально
				// результат
				double resOutput = solve(InputX, InputY);
				// записываем в поле вывода
				this->txtOut->Text = System::Convert::ToString(resOutput);
			}
		}
	}
\end{minted}
\section{Фрагменты кода программы <<Рекурсивные вычисления>>}\label{app:task3}

Функция очистки текстовых полей в форме:

\begin{minted}{c++}
	private: void ClearAll() { // очистка полей
		this->txtOut->Text = "";
		errPrX->SetError(txtInX, String::Empty);
		errPrN->SetError(txtInN , String::Empty);
	}
\end{minted}

Нажатие кнопки <<Вычислить>> установлено выполнение следующего кода:

\begin{minted}{c++}
	private: System::Void btnStart_Click(System::Object^ sender, System::EventArgs^ e) {
		ClearAll();
		double InputX;
		long long InputN;
		// переводим сторку из TextBox в число
		bool parseX = double::TryParse(this->txtInX->Text, InputX);
		bool parseN = Int64::TryParse(this->txtInN->Text, InputN);
		// ввели не число
		if (!parseX) {
			errPrX->SetError(txtInX, "X не число");
			if (!parseN) {
				errPrN->SetError(txtInN, "N не целое число");
			}
		}
		else if (!parseN) {
			errPrN->SetError(txtInN, "N не целое число");
		}
		else {
			// все нормально 
			double result;
			// вычисляем результат
			result = recursion(InputX, InputN);
			// записываем в поле вывода
			this->txtOut->Text = System::Convert::ToString(result);
		}
	}
\end{minted}

\section{Фрагменты кода программы <<Обработка табличных данных. Часть 1.>>}\label{app:task4}

Нажатие кнопки <<Сумма>> установлено выполнение следующего кода:

\begin{minted}{c++}
	// среднее арифметическое
	private: System::Void SumBtn_Click(System::Object^ sender, System::EventArgs^ e) {
		ClearAll();
		long long InputA;
		long long InputB;
		// переводим сторку из TextBox в число
		bool parseA = Int64::TryParse(this->ATxtBox->Text, InputA);
		bool parseB = Int64::TryParse(this->BTxtBox->Text, InputB);
        // проверки
		if (!parseA) {
			errPrA->SetError(ATxtBox, "A не число");
			if (!parseB) {
				errPrB->SetError(BTxtBox, "B не число");
			}
		}
		else if (!parseB) {
			errPrB->SetError(BTxtBox, "B не число");
		}
		else if (InputA < 0) {
			errPrA->SetError(ATxtBox, "А не может быть отрицательным");
		}
		else if (InputB > ArDtGr->RowCount -1) {
			errPrB->SetError(BTxtBox, "B больше чем кол-во элементов");
		}
		else {
			// подсчет среднего арифметического
			long long countres = 0;
			long long count = 0;
			double res = 0;
			double sum = 0;
			for (long long i = 0; i <= ArDtGr->RowCount - 1; i++) {
				if ((long long)ArDtGr->Rows[i]->Cells[0]->Value % 2 != 0) {
					res += (long long)ArDtGr->Rows[i]->Cells[0]->Value;
					countres++;
				}
			}
			for (long long i = InputA; i <= InputB; i++) {
				if ((long long)ArDtGr->Rows[i]->Cells[0]->Value % 2 != 0) {
					sum += (long long)ArDtGr->Rows[i]->Cells[0]->Value;
					count++;
				}
			}

			res -= sum;
			countres -= count;

			// обработка случая, где таких элементов нет
			if (res != 0 && countres != 0) {
				res = (res) / countres;
				this->SumTxtBox->Text = res.ToString();
			}
			else {
				this->SumTxtBox->Text = "Таких элементов нет";
			}
		}
	}
\end{minted}

Нажатие кнопки <<Удалить>> установлено выполнение следующего кода:

\begin{minted}{c++}
	// удаление строки
	private: System::Void DelBtn_Click(System::Object^ sender, System::EventArgs^ e) {
        // очищаем поля
		ClearAll();
        // Если строка не последняя, то удаляем строку
		if (ArDtGr->RowCount != 0) {
			ArDtGr->Rows->RemoveAt(ArDtGr->RowCount - 1);
		}
	}
\end{minted}

\section{Фрагменты кода программы <<Обработка табличных данных. Часть 2.>>}\label{app:task5}

Нажатие кнопки <<Удалить столбец>> установлено выполнение следующего кода:

\begin{minted}{c++}
	// Удаление столбца
	private: System::Void btnRemoveColumn_Click(System::Object^ sender, System::EventArgs^ e) {
		ClearAll();
		if (countColumn == 1) {
			erZeroColumn->SetError(btnRemoveColumn, "Нельзя удалить этот столбец");
			return;
		}
		dataGridInput->Columns->Remove("in" + System::Convert::ToString(countColumn));
		dataGridOutput->Columns->Remove("out" + System::Convert::ToString(countColumn));
		// Уменьшаем счетчик, если удалили столбец
		--countColumn;
	}
\end{minted}

Нажатие кнопки <<Заменить>> установлено выполнение следующего кода:

\begin{minted}{c++}
	private: System::Void btnStart_Click(System::Object^ sender, System::EventArgs^ e) {
		ClearAll();
		if (countRow < 2) {
			erChanges->SetError(btnStart, "Для выполнения изменений нужно 2 и более строк");
			return;
		}

		// Условно берем строку по индексу 1 как опорный максимум
		int indexMaxRow = 1;
		int sumMax = 0;
		for (int i = 0; i < countColumn; ++i) {
			int cellValue = 0;
			bool cell = Int32::TryParse(System::Convert::ToString(dataGridInput->Rows[0]->Cells[i]->Value), cellValue);
			if (!cell) {
				erChanges->SetError(btnStart, "Не число");
				return;
			}
			sumMax += cellValue;
			dataGridOutput->Rows[0]->Cells[i]->Value = System::Convert::ToString(cellValue);
		}

		// Находим максимальную строку
		for (int i = 1; i < countRow; ++i) {
			int sum = 0;
			for (int j = 0; j < countColumn; ++j) {
				int cellValue = 0;
				bool cell = Int32::TryParse(System::Convert::ToString(dataGridInput->Rows[i]->Cells[j]->Value), cellValue);
				if (!cell) {
					erChanges->SetError(btnStart, "Не число");
					return;
				}
				sum += cellValue;
				dataGridOutput->Rows[i]->Cells[j]->Value = System::Convert::ToString(cellValue);
			}
			if (sum > sumMax) {
				indexMaxRow = i;
				sumMax = sum;
			}
		}

		// Меняем 0 строку с максимальной
		for (int i = 0; i < countColumn; ++i) {
			int cellValue = 0;
			int cellMaxValue = 0;
			bool cell = Int32::TryParse(System::Convert::ToString(dataGridInput->Rows[0]->Cells[i]->Value), cellValue);
			bool cellMax = Int32::TryParse(System::Convert::ToString(dataGridInput->Rows[indexMaxRow]->Cells[i]->Value), cellMaxValue);
			if (!cell || !cellMax) {
				erChanges->SetError(btnStart, "Не число");
				return;
			}
			dataGridOutput->Rows[indexMaxRow]->Cells[i]->Value = System::Convert::ToString(cellValue);
			dataGridOutput->Rows[0]->Cells[i]->Value = System::Convert::ToString(cellMaxValue);
		}
	}
\end{minted}
\section{Фрагменты кода программы <<Матричный калькулятор>>}\label{app:task6}

Функция нахождения ранга:

\begin{minted}{c++}
		// Нахождение ранга
		int Rank(int** matrix, int row, int column) {
			if (row > column) {
				column = row;
			}
			else {
				row = column;
			}

			// Заполняем доп матрицу, если она не квадратная изначально
			// то добавляем недостающее кол-во строк/столбцов с элементами = 0
			double** tempMatrix = new double* [row];
			for (int i = 0; i < row; ++i) {
				tempMatrix[i] = new double[row];
				for (int j = 0; j < column; ++j) {
					if (i >= matrixA->RowCount) {
						tempMatrix[i][j] = 0;
					}
					else if (j >= matrixA->ColumnCount) {
						tempMatrix[i][j] = 0;
					}
					else {
						tempMatrix[i][j] = matrix[i][j];
					}
				}
			}

			int rank = row;
			for (int row_i = 0; row_i < rank; ++row_i) {
				// Если диагональный элемент != 0
				if (tempMatrix[row_i][row_i]) {
					for (int column_i = 0; column_i < column; ++column_i) {
						if (column_i != row_i) {
							double mult = tempMatrix[column_i][row_i] /
								tempMatrix[row_i][row_i];
							for (int i = 0; i < rank; ++i) {
								tempMatrix[column_i][i] -= mult * tempMatrix[row_i][i];
							}
						}
					}
				}
				//Если диагональный элемент == 0, то
				else {
					bool reduce = true;
					// находим не нулевой элемент в данном столбце
					for (int i = row_i + 1; i < column; ++i) {
						// Меняем строку с ненулевым элементом с данной строкой
						if (tempMatrix[i][row_i]) {
							double* temp = tempMatrix[row_i];
							tempMatrix[row_i] = tempMatrix[i];
							tempMatrix[i] = temp;
							reduce = false;
							break;
						}
					}
					// Если не нашли не один ненулевой элемент в данном столбце
					// то все элементы этого столбца == 0
					if (reduce) {
						// уменьшаем ранг
						rank--;
						for (int i = 0; i < column; ++i) {
							tempMatrix[i][row_i] = tempMatrix[i][rank];
						}
					}
					// Прогоняем еще раз эту строку
					row_i--;
				}
			}
			return rank;
		}
\end{minted}

Функция умножения матрицы A на число num:

\begin{minted}{c++}
		// умножение матрицы A на число num
		void MultNum(int row, int column, int num) {
			int a;
			bool check;
			for (int i = 0; i < row; i++) {
				for (int j = 0; j < column; j++) {
					// Проверка на не целое число
					check = Int32::TryParse(System::Convert::ToString(matrixA->Rows[i]->Cells[j]->Value), a);
					if (!check) {
						throw gcnew FormatException("В матрице присутствуют не целые числа");
					}
					// Умножение на число
					matrixResult->Rows[i]->Cells[j]->Value = a * num;
				}
			}
		}
\end{minted}

\section{Фрагменты кода программы <<Использование коллекций>>}\label{app:task7}

Нажатие кнопки <<Новая очередь (новый элемент после всех макс)>> установлено выполнение следующего кода:

\begin{minted}{c++}
		private: System::Void btnNewAfterMax_Click(System::Object^ sender, System::EventArgs^ e) {
			ClearEP();
			this->tBOutputNAM->Clear();
			// Вспомогательная очередь
			System::Collections::Generic::Queue<int> buffer;
			System::Collections::Generic::Queue<int> buffer2;
			if (!q.Count) {
				this->eP1->SetError(btnNewAfterMax, "Пустая очередь");
				return;
			}
			int number, max = q.Peek();
			bool res = Int32::TryParse(tBInputNAM->Text, number);
			if (!res) {
				this->eP1->SetError(tBInputNAM, "Не целое число");
				return;
			}
			// Пока очередь не пуста
			while (q.Count) {
				// Записываем во вспомогательную очередь первый элемент
				buffer2.Enqueue(q.Peek());
				// Находим максимум
				if (q.Peek() > max) {
					max = q.Peek();
				}
				// Удаляем первый элемент очереди
				q.Dequeue();
			}
			// Пока очередь не пуста
			while (buffer2.Count) {
				// Записываем в основную очередь первый элемент
				q.Enqueue(buffer2.Peek());
				// Удаляем первый элемент очереди
				buffer2.Dequeue();
			}
			// Пока очередь не пуста
			while (q.Count) {
				// Записываем во вспомогательную очередь первый элемент
				buffer.Enqueue(q.Peek());
				// Записываем после максимальных элементов новый элемент
				if (q.Peek() == max) {
					buffer.Enqueue(number);
				}
				buffer2.Enqueue(q.Peek());
				// Удаляем первый элемент очереди
				q.Dequeue();
			}
			String^ result = "";
			// Пока очередь не пуста
			while (buffer2.Count) {
				// Записываем в основную очередь первый элемент
				q.Enqueue(buffer2.Peek());
				// Удаляем первый элемент очереди
				buffer2.Dequeue();
			}
			while (buffer.Count) {
				// Записываем результат
				result += System::Convert::ToString(buffer.Peek() + " ");
				// Удаляем первый элемент очереди
				buffer.Dequeue();
			}
			this->tBOutputNAM->Text = result;
		}
\end{minted}

Нажатие кнопки <<Сумма четных в интервале>> установлено выполнение следующего кода:

\begin{minted}{c++}
		private: System::Void btnSum_Click(System::Object^ sender, System::EventArgs^ e) {
			ClearEP();
			// Вспомогательная очередь
			System::Collections::Generic::Queue<int> buffer;
			int a, b;
			bool resA = Int32::TryParse(tBInputA->Text, a);
			bool resB = Int32::TryParse(tBInputB->Text, b);
			int sum = 0, i = 0;
			if (!resA) {
				this->eP1->SetError(tBInputA, "Не целое число");
				return;
			}
			if (!resB) {
				this->eP1->SetError(tBInputB, "Не целое число");
				return;
			}
			if (a < 0 || a > b || a > q.Count) {
				this->eP1->SetError(tBInputA, "Не верный интервал");
				return;
			}
			if (b > q.Count) {
				this->eP1->SetError(tBInputB, "Не верный интервал");
				return;
			}

			// Пока очередь не пуста
			while (q.Count) {
				// Записываем во вспомогательную очередь первый элемент
				buffer.Enqueue(q.Peek());
				// Суммируем четные элементы
				if (q.Peek() % 2 == 0 && (i >= a && i <= b)) {
					sum += q.Peek();
				}
				// Удаляем первый элемент очереди
				q.Dequeue();
				i++;
			}
			// Пока очередь не пуста
			while (buffer.Count) {
				// Записываем в основную очередь первый элемент
				q.Enqueue(buffer.Peek());
				// Удаляем первый элемент очереди
				buffer.Dequeue();
			}
			// Выводим результат
			this->tBOutputSum->Text = Convert::ToString(sum);
		}
\end{minted}

\section{Фрагменты кода программы <<Работа с файлами>>}\label{app:task8}

Нажатие кнопки <<Открыть файл>> установлено выполнение следующего кода:

\begin{minted}{c++}
		// Чтение данных
		private: System::Void btnReadFile_Click(System::Object^ sender, System::EventArgs^ e) {
			this->errPr->SetError(this->dGrInput, System::String::Empty);
			System::IO::Stream^ myStream;
			btnSaveInFile->Enabled = true;
			if (this->openFile->ShowDialog() == System::Windows::Forms::DialogResult::OK) {
				if ((myStream = openFile->OpenFile()) != nullptr) {
					System::IO::StreamReader^ sw =
						gcnew System::IO::StreamReader(myStream,
							System::Text::Encoding::GetEncoding(1251)
						);

					// Очистка таблиц
					ClearTables();

					System::String^ strBuf = "";
					array<System::String^>^ arrBuf;
					array<System::String^>^ dateBuf;
					int i = 0;
					// Вывод в таблицы
					while (sw->Peek() + 1) {
						DateTime dateTemp;
						this->dGrInput->Rows->Add(1);
						// Чтение строки из файла
						strBuf = sw->ReadLine();
						try {
							// Вывод ФИО в таблицу dGrInput
							arrBuf = strBuf->Split(' ');
							this->dGrInput->Rows[i]->Cells[0]->Value = arrBuf[0];
							this->dGrInput->Rows[i]->Cells[1]->Value = arrBuf[1];
							this->dGrInput->Rows[i]->Cells[2]->Value = arrBuf[2];

							// Проверка на корректность даты
							dateBuf = arrBuf[3]->Split('.');
							strBuf = dateBuf[0] + "." + dateBuf[1] + "." + dateBuf[2];
							bool res = DateTime::TryParse(strBuf, dateTemp);
							if (!res) {
								this->errPr->SetError(this->dGrInput, "Не корректная дата");
								// Очистка таблиц
								ClearTables();
								return;
							}
							// Вывод даты в таблицу dGrInput
							this->dGrInput->Rows[i]->Cells[3]->Value = dateTemp.Day + "." + dateTemp.Month + "." + dateTemp.Year;
							int grade;
							bool flag = true;
							// Вывод оценок в таблицу dGrInput
							for (int j = 4; j < 9; ++j) {
								bool res = Int32::TryParse(arrBuf[j], grade);

								if (!res || grade > 5 || grade < 2) {
									this->errPr->SetError(this->dGrInput, "Не корректные оценки");
									// Очистка таблиц
									ClearTables();
									return;
								}
								else if (grade <= 5 && grade >= 2) {
									this->dGrInput->Rows[i]->Cells[4]->Value += " " + System::Convert::ToString(grade);
								}
								if (grade == 2) {
									// Если есть 2 в оценках, то flag = false
									flag = false;
								}
							}
							// Если flag == false => есть хотя бы одна 2 => его в таблицу результата выводить не нужно
							if (!flag) {
								++i;
								continue;
							}
							// Вывод в таблицу dGrOutput сдавших сессию
							this->dGrOutput->Rows->Add(1);
							for (int j = 0; j < 5; ++j)
								this->dGrOutput->Rows[
									this->dGrOutput->RowCount - 1
								]->Cells[j]->Value = this->dGrInput->Rows[i]->Cells[j]->Value;
						}
						catch (...) {
						}
						++i;
					}

				}
			}

		}
\end{minted}

Нажатие кнопки <<Удалить строку>> установлено выполнение следующего кода:

\begin{minted}{c++}
		// Удаление строки
		private: System::Void btnRemoveRow_Click(System::Object^ sender, System::EventArgs^ e) {
			this->errPr->SetError(this->dGrOutput, System::String::Empty);
			if (!this->dGrOutput->CurrentRow) {
				this->errPr->SetError(this->dGrOutput, "Нельзя удалить несуществующую строку");
				return;
			}
			if (!this->dGrOutput->CurrentRow->IsNewRow) {
				int i = this->dGrOutput->CurrentRow->Index;
				this->dGrOutput->Rows->Remove(this->dGrOutput->Rows[i]);
			}
		}
\end{minted}
\section{Фрагменты кода программы <<Тест>>}\label{app:task9}

Нажатие кнопки <<Далее>> установлено выполнение следующего кода:

\begin{minted}{c++}
		System::Void actionBtn_Click(System::Object^ sender, System::EventArgs^ e) {
			errPr->Clear();
			std::string answer = "";
			// Считываем ответ на вопрос
			if (iter != Questions->size()) {
				answer = ReadAnswer();
			}
			// если ответ правильного формата
			if (answer != "") {
				int i = RandIndex->at(iter);
				bool flag = false;
				// проверка ответа
				if (Questions->at(i).getRightAnswer() == answer) {
					// выставление баллов за ответ
					Questions->at(i).setResult(1);
					flag = true;
				}
				else {
					// выставление баллов за ответ
					Questions->at(i).setResult(0);
				}
				// Вывод результата ответа на данный вопрос (правильно ответили или нет)
				MessageForm^ msgForm = gcnew MessageForm();
				msgForm->setFlag(flag);
				msgForm->ShowDialog();
				iter++;
				// Переход к следующему вопросу
				if (iter != Questions->size()) {
					countBox->Text = Convert::ToString(iter + 1);
					ChooseForm();
					int i = RandIndex->at(iter);
					questBox->Text = gcnew String(Questions->at(i).getText().c_str());
				}
				// Если вопросы кончились выводим результат
				else {
					questBox->Text = gcnew String("Вы набрали " + CalcResult() + " баллов");
				}
			}
		}
\end{minted}

Функция подсчета правильных ответов:

\begin{minted}{c++}
		// Подсчет правильных ответов
		int CalcResult() {
			int sum = 0;
			// Проходимся по вектору вопросов
			for (int i = 0; i < Questions->size(); ++i) {
				// Прибавляем число из результата вопроса (1 или 0)
				sum += Questions->at(i).getResult();
			}
			// Выводим кол-во правильных ответов
			return sum;
		}
\end{minted}
